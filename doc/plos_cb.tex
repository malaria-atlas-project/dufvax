% Template for PLoS
% Version 1.0 January 2009
%
% To compile to pdf, run:
% latex plos.template
% bibtex plos.template
% latex plos.template
% latex plos.template
% dvipdf plos.template

\documentclass[10pt]{article}

% FIXME Remove this
\usepackage{garamond}
\usepackage[T1]{fontenc}
% FIXME end

\usepackage{epsfig} 

% amsmath package, useful for mathematical formulas
\usepackage{amsmath}
% amssymb package, useful for mathematical symbols
\usepackage{amssymb}

% graphicx package, useful for including eps and pdf graphics
% include graphics with the command \includegraphics
\usepackage{graphicx}

% cite package, to clean up citations in the main text. Do not remove.
\usepackage{cite}

\usepackage{color} 

% Use doublespacing - comment out for single spacing
%\usepackage{setspace} 
%\doublespacing


% Text layout
\topmargin 0.0cm
\oddsidemargin 0.5cm
\evensidemargin 0.5cm
\textwidth 16cm 
\textheight 21cm

% Bold the 'Figure #' in the caption and separate it with a period
% Captions will be left justified
\usepackage[labelfont=bf,labelsep=period,justification=raggedright]{caption}

% Use the PLoS provided bibtex style
\bibliographystyle{plos2009}

% Remove brackets from numbering in List of References
\makeatletter
\renewcommand{\@biblabel}[1]{\quad#1.}
\makeatother


% Leave date blank
\date{}

\pagestyle{myheadings}
%% ** EDIT HERE **


%% ** EDIT HERE **
%% PLEASE INCLUDE ALL MACROS BELOW

%% END MACROS SECTION

\begin{document}

% FIXME: Remove this
\garamond

% Title must be 150 characters or less
\begin{flushleft}
{\Large
\textbf{Spatiotemporal prediction of Duffy negativity and \emph{Plasmodium vivax} infection prevalence}
}
% Insert Author names, affiliations and corresponding author email.
\\
Anand P. Patil$^{\ast}$,  
Rosalind E. Howes,
Carlos A. Guerra,
Peter W. Gething,
Simon I. Hay
\\
Malaria Atlas Project, Spatial Ecology and Epidemiology Group, University of Oxford, Oxford, UK
\\

% \bf{3} Author3 Dept/Program/Center, Institution Name, City, State, Country
% \\
$\ast$ E-mail: anand.patil@zoo.ox.ac.uk
\end{flushleft}

% Please keep the abstract between 250 and 300 words
\section*{Abstract}

The authors present a Bayesian spatial analysis of Duffy negativity and \emph{P. vivax} endemicity in trans-Sahelian Africa. The Duffy negativity dataset contains a mixture of records derived from three different assays of blood-group phenotype, as well as two different assays of genotype. The \emph{P. vivax} dataset contains records of endemicity, the times at which surveys were performed and the ages of the survey participants. The statistical model links all of these data sources in a coherent, scientifically defensible manner. The predictions of \emph{P. vivax} endemicity and Duffy negativity are informed by both datasets, resulting in improved accuracy and precision relative to what is possible using either dataset alone. The coherency of the model makes it possible to generate a wide variety of predictions, with fully quantified uncertainty.

% Please keep the Author Summary between 150 and 200 words
% Use first person. PLoS ONE authors please skip this step. 
% Author Summary not valid for PLoS ONE submissions.   
\section*{Author Summary}

Issues in public health and epidemiology are complex and interrelated. Bayesian statistics is a statistical paradigm that makes it possible incorporate data from many disparate sources in a sensible way. In this paper, we combine two very different but related datasets in the same Bayesian geographical-statistical analysis. \emph{Plasmodium vivax} is a strain of human malaria, and Duffy negativity is a genetic trait that protects people from it. By combining data on the prevalence of each, we are able to produce better predictions of \emph{P. vivax} infection prevalence and Duffy negativity prevalence than we could using either source of data in isolation.

\section*{Introduction}

\emph{P. vivax} is increasingly being recognized as a tropical disease with serious public health implications. \textbf{Carlos/ Simon: some epidemiological context}

Duffy negativity is a genetic trait known to be very strongly related to \emph{P. vivax} in the sense that carriers (with few exceptions) are refractory to \emph{P. vivax} infection. \textbf{Ros: Brief introduction to Duffy and Duffy/vivax interaction}

\medskip
The Malaria Atlas Project (MAP) has assembled a comprehensive global database of Duffy blood group records \cite{howes:2010}, and is in the process of assembling a global database of observations of \emph{P. vivax} endemicity. The data abstraction protocols for the latter are described in \textbf{cite}. The eventual purpose of both data collection efforts is to produce evidence-based global predictions of \emph{P. vivax} endemicity. For the purpose of methodological exposition, in this paper we focus on the subset of both datasets in trans-Sahelian Africa (figure \ref{fig:datasets}). The spatial distributions of the two datasets suggest that the Duffy negativity data are a useful addition. The available \emph{P. vivax} data are highly concentrated in Ethiopia and the Sudan, while the Duffy dataset has much better coverage of West Africa.

\begin{figure}
    \begin{center}
        \epsfig{file=figs/figs/datasets.pdf, width=16cm} 
        \caption{The distributions of \emph{P. vivax} (left) and Duffy blood group (right) surveys used in the analysis. All of the \emph{P. vivax} datapoints represent community-based prevalence surveys \textbf{CITE}. The full Duffy database presented by Howes et al. \cite{howes:2010} contains five types of records, but the window considered in the current study contains three. \textbf{Description of gen, phe and aphe datatypes from Ros.}} 
    \label{fig:datasets}
    \end{center}    
\end{figure}

Combining the two datasets is not straightforward using standard geostatistical methods. \textbf{Pete please make sure I'm talking sense here.} It is not appropriate to model Duffy negativity as a linear predictor of \emph{P. vivax} endemicity \cite{goovaerts:1997,wackernagel:2003} or its log-odds ratio \cite{banerjee:2004,mccullagh:1999}. The simple mathematical relationship between Duffy negativity $p_0$ and \emph{P. vivax} endemicity $p_v$ is that $p_v\le 1-p_0$ everywhere. This relationship is inconvenient to capture in the linear model and generalized linear model frameworks. Similar considerations make standard multivariate methods such as cokriging \cite{wackernagel:2003} and a spatial version of the multivariate GLM \cite{mccullagh:1999} for multinomial data seem to be the wrong tools for the job.

\medskip
On the other hand, it is quite straightforward to combine the datasets in a Bayesian hierarchical model. The development of the model follows the scientific reasoning closely. First, two standard spatial generalized linear models are constructed to describe the respective frequencies of the genetic variants at the two loci that determine Duffy blood group. These frequencies are related to the heterogeneous records in the Duffy negativity database using simple population genetics. Then, a spatiotemporal, seasonal GLM along the lines of Hay et al. \cite{hay:2007}  is constructed to describe \emph{P. vivax} endemicity within non-Duffy negative individuals.

The model development proceeds in the direction that makes scientific sense: \emph{P. vivax} endemicity is mediated by Duffy negativity prevalence over intra-generational timescales. However, information `flows in both directions' in that the \emph{P. vivax} observations provide information about Duffy negativity as well as the converse. Analyzing the datasets together results in better-informed predictions for both attributes. 

\documentclass[a4paper]{article}
\usepackage{fullpage}
\usepackage{epsfig}
\usepackage{pdfsync} 
\usepackage{amsfonts}
\usepackage{amsmath} 
\usepackage{garamond}
\usepackage[colorlinks]{hyperref}
\usepackage[T1]{fontenc} 
\begin{document}

\garamond
\title{Methods SI for duffy paper}
\author{Anand Patil}
\maketitle

\section{License} % (fold)
\label{sec:license}
This document is licensed under the Creative Commons attribution share-alike license, see \texttt{LICENSE} in the root directory. Copyright \copyright\ 2009 Anand Patil
% section license (end)

\section{Statistical analysis} 
\subsection{Random fields}
The model targets two spatially-varying allele frequencies, the frequency of the Fyb mutation and the frequency of the silencing mutation in the promoter region within Fyb individuals. These are denoted $p_{ab}(x)$ and $p_0(x)$ respectively, where $x$ is a location. A third allele frequency, the frequency of the silencing mutation within Fya individuals, is modeled as a small constant denoted $p_1$. The model for these allele frequencies is as follows:
\begin{eqnarray*}
    m_{ab} \sim \textup{Normal}(0,10000) \\
    m_0 \sim \textup{Normal}(0,10000)\\
    \beta_{\mathtt{africa}} \sim \textup{Normal}(0,10000)\\
    p_1\sim\textup{Uniform}(0,.4)\\
    \\
    M_{ab}(x) = \beta_{\textup{africa}} 1_{x\textup{\ in\ africa}} + m_{ab}\\
    M_0(x) = m_0\\
    \\
    \phi_{ab}\sim\textup{Exponential}(.1)\\
    \phi_{0}\sim\textup{Exponential}(.1)\\
    \theta_{ab}\sim\textup{Exponential}(.1)\\
    \theta_{0}\sim\textup{Exponential}(.1)\\
    \nu_{ab}\sim\textup{Uniform}(0,3)\\
    \nu_{0}\sim\textup{Uniform}(0,3)\\
    V_{ab}\sim\textup{Exponential}(.1)\\
    V_0\sim\textup{Exponential}(.1)\\
    \\
    C_{ab}(x,y) = \phi_{ab}\textup{Mat\`ern}(d(x,y)/\theta_{ab};\nu_{ab})+V_{ab}1_{x=y}\\
    C_{0}(x,y) = \phi_{0}\textup{Mat\`ern}(d(x,y)/\theta_{0};\nu_{0})+V_{0}1_{x=y}\\
    \\
    f_{ab}\sim\textup{GP}(M_{ab},C_{ab})\\
    f_{0}\sim\textup{GP}(M_{0},C_{0})\\
    p_{ab}(x)=\textup{logit}^{-1}(f_{ab}(x)) \\
    p_{0}(x)=\textup{logit}^{-1}(f_{0}(x))     
\end{eqnarray*}
The mean function for $f_0$ simply returns a constant, $m_0$. The mean of $f_{ab}$ takes presence in Africa as a covariate with coefficient $\beta_{\textup{africa}}$, and also a constant term $m_{ab}$. 

Both fields use the Mat\`ern covariance function [cite Banerjee]. The range parameters of $f_0$ and $f_{ab}$ are $\theta_0$ and $\theta_{ab}$, respectively; the corresponding amplitude parameters are $\phi_0$ and $\phi_{ab}$; the degree-of-differentiability parameters are denoted $\nu_{0}$ and $\nu_{ab}$; and the nugget variances are $V_0$ and $V_{ab}$. The distance function $d$ gives the great-circle distance between its arguments.

The Gaussian random fields are converted to probabilities using the standard inverse logit link function.

\subsection{Likelihood} % (fold)
\label{sec:likelihood}

The a/b switch mutation happens with probability $P(b=1)=p_{ab}(x)$, which should be much higher for $x$ in Africa. Given that $b=1$, the silencing mutation in the promoter region happens with probability $p_0(x)$. Given that $b=0$, the silencing mutation happens with probability $p_1$, which is assumed to be a small, constant value. Hardy-Weinberg assumptions apply to the genotype frequencies.
\begin{description}
    % \item[lon,lat]: Standard 
    % \item[n]: Sample size
    % \item[africa]: Whether the point was taken in Africa
    % \item[data]: The type of data
    \item[gen*]: Genotype data. The haplotype frequencies are:
    \begin{description}
        \item[gena] $(1-p_{ab}(x))(1-p_1)$
        \item[genb] $p_{ab}(x)(1-p_0(x))$ 
        \item[gen0] $p_{ab}(x)p_0(x)$
        \item[gen1] $(1-p_{ab}(x))$
    \end{description}
    The genotype frequencies (genaa, genab, etc.) can be obtained using the standard Hardy-Weinberg formulas. For example, the frequency of $genab$ is twice the product of the frequencies of $gena$ and $genb$, which is $2(1-p_{ab}(x))(1-p_1)p_{ab}(x)(1-p_0(x))$
    \item[phe*]: Phenotype data.
    \begin{description}
        \item[pheab] This can only happen if the genotype is genab.
        \item[phea] This can happen if the genotype is gena0, gena1 or genaa.
        \item[pheb] This can happen if the genotype is genb0, genb1 or genbb.
        \item[phe0] This can only happen if the phenotype is gen00, gen01 or gen11.
    \end{description}
    \item[aphe*]: Phenotype data, Fya+/- only.
    \begin{description}
        \item[aphea] This can happen if the genotype is genaa, genab, gena1 or gena0.
        \item[aphe0] The complement of aphea.
    \end{description}
    \item[bphe*] Phenotype data, Fyb+/- only.
    \begin{description}
        \item[bpheb] This can happen if the genotype is genbb, genab, genb0 or genb1.
        \item[bphe0] The complement of bpheb. 
    \end{description}
    \item[prom]: Molecular study looked only at promoter region.
    \begin{description}
        \item[prom0]: This can only happen if the genotype is gen00, gen01 or gen11.
        \item[promab]: The complement of pos0.
    \end{description}
\end{description}

The sampling distributions are assumed to be multinomial conditional on the appropriate individual phenotype or genotype probabilities described above. This likelihood completes the Bayesian probability model.

\subsection{Implementation} 
The model was implemented in the \href{www.python.org}{Python} programming language. The code is available from \href{github.com/malaria-atlas-project/ibd-world}{MAP's code repository}. It was fitted with the Markov chain Monte Carlo algorithm [cite Gilks] using the open-source Bayesian analysis package \href{code.google.com/p/pymc}{PyMC}. Dynamic traces [cite Gilks] are available upon request.

Maps were generated using optimized Python code available from \href{github.com/malaria-atlas-project/generic-mbg}{MAP's code repository}.


\end{document}
% Results and Discussion can be combined.
\section*{Results}

Make maps and possibly UUorld thingies.

\subsection*{Duffy negativity data only}

\subsection*{All Duffy data}

\subsection*{\emph{P. vivax} data only}

\subsection*{All available data}
 

\section*{Discussion}



% Do NOT remove this, even if you are not including acknowledgments
\section*{Acknowledgments}


%\section*{References}
% The bibtex filename
\bibliography{plos_cb}

\section*{Figure Legends}
%\begin{figure}[!ht]
%\begin{center}
%%\includegraphics[width=4in]{figure_name.2.eps}
%\end{center}
%\caption{
%{\bf Bold the first sentence.}  Rest of figure 2  caption.  Caption 
%should be left justified, as specified by the options to the caption 
%package.
%}
%\label{Figure_label}
%\end{figure}


\section*{Tables}
%\begin{table}[!ht]
%\caption{
%\bf{Table title}}
%\begin{tabular}{|c|c|c|}
%table information
%\end{tabular}
%\begin{flushleft}Table caption
%\end{flushleft}
%\label{tab:label}
% \end{table}

\end{document}

