\documentclass[a4paper]{article}
\usepackage{fullpage}
\usepackage{epsfig}
\usepackage{pdfsync} 
\usepackage{amsfonts}
\usepackage{amsmath} 
\begin{document}

\title{Notes on the Duffy model}
\author{Anand Patil}
\maketitle

\section{License} % (fold)
\label{sec:license}
This document is licensed under the Creative Commons attribution share-alike license, see \texttt{LICENSE} in the root directory. Copyright \copyright\ 2009 Anand Patil
% section license (end)

\section{Likelihood} % (fold)
\label{sec:likelihood}

The a/b switch mutation happens with probability $P(b=1)=p_{ab}(x)$, which should be much higher for $x$ in Africa. Given that $b=1$, the silencing mutation in the promoter region happens with probability $p_0(x)$. Hardy-Weinberg assumptions apply to the genotype frequencies.
\begin{description}
    \item[lon,lat]: Standard 
    \item[n]: Sample size
    \item[africa]: Whether the point was taken in Africa
    \item[data]: The type of data
    \item[gen*]: Genotype data. Probabilities are:
    \begin{description}
        \item[genaa] $(1-p_{ab}(x))^2$
        \item[genab] $2(1-p_{ab}(x))p_{ab}(x)(1-p_0(x))$ 
        \item[gen00] $[p_{ab}(x)p_0(x)]^2$
        \item[gena0] $2(1-p_{ab}(x))p_{ab}(x)p_0(x)$
        \item[genb0] $2p_{ab}(x)^2(1-p_0(x))p_0(x)$
        \item[genbb] $[p_{ab}(x)(1-p_0(x))]^2$
    \end{description}
    \item[phe*]: Phenotype data. Probabilities are:
    
    \begin{description}
        \item[pheab] This can only happen if the genotype is genab, so the probability is $2(1-p_{ab}(x))p_{ab}(x)(1-p_0(x))$
        \item[phea] This can happen if the genotype is gena0 or genaa, so the probability is $2(1-p_{ab}(x))p_{ab}(x)p_0(x)+(1-p_{ab}(x))^2$
        \item[pheb] This can happen if the genotype is genb0 or genbb, so the probability is $2p_{ab}(x)^2(1-p_0(x))p_0(x)+[p_{ab}(x)(1-p_0(x))]^2$
        \item[phe0] This can only happen if the phenotype is gen00, so the probability is $[p_{ab}(x)p_0(x)]^2$
    \end{description}
    \item[aphe*]: Phenotype data, Fya+/- only. Probabilities are:
    \begin{description}
        \item[aphea] This can happen if the genotype is genaa, genab or gena0. The probability is $(1-p_{ab}(x))^2+2(1-p_{ab}(x))p_{ab}(x)(1-p_0(x))+2(1-p_{ab}(x))p_{ab}(x)p_0(x)$.
        \item[aphe0] This can happen if the genotype is gen00, genb0 or genbb. The probability is $[p_{ab}(x)p_0(x)]^2+2p_{ab}(x)^2(1-p_0(x))p_0(x)+[p_{ab}(x)(1-p_0(x))]^2$.
    \end{description}
    \item[gf*]: Allele frequencies calculated by authors, raw data not available. Not bothering with likelihoods for these yet.
    \item[pos]: Molecular study looked only at promoter region. Probabilities are:
    \begin{description}
        \item[pos0]: This can only happen if the genotype is gen00, so the probability is $[p_{ab}(x)p_0(x)]^2$.
        \item[negab]: The complement of pos0, so the probability is $1-[p_{ab}(x)p_0(x)]^2$.
    \end{description}
\end{description}


\end{document}